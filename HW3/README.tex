\documentclass[10pt]{article}

\usepackage{enumerate}
\usepackage{amsmath}
\usepackage[top=1in, bottom=1in, left=1in, right=1in]{geometry}

\title{600.465 -- Natural Language Processing\\
Assignment 2: Probability and Vector Exercises}
\author{}
\date{February 2016}

\begin{document}

\maketitle
\begin{enumerate}
    \item  % problem 1
    Sample 1:\\ $log_2$probability: $-12111.3$, word count: $1686$, perplexity per word: $2^{12111.3/1686} \approx 145.37$\\
    Sample 2:\\ $log_2$probability: $-7388.84$, word count: $978$, perplexity per word: $2^{7388.84/978} \approx 188.06$\\
    Sample 3:\\ $log_2$probability: $-7468.29$, word count: $985$, perplexity per word: $2^{7468.29/985} \approx 191.61$\\
    
    When switch to the larger {\tt switchboard} corpus the $log_2$probabilities go slightly lower while the perplexities go up a lot for they are calculated by taking exponential . This is because typically larger corpus have more words than smaller ones, making the probabilities of words in the sample have lower probabilities to appear. 
    
     \addtocounter{enumi}{1}
     \item % problem 3
     	\begin{enumerate}
     		\item
     		We chose the language ID problem. The lowest error rate we can achieve is $0.933$.
     		\item
     		The value of $\lambda$ we use is $2.7$.
     		\item
     		Test result for english:\\
     		\texttt{342 looked more like en.1K (92.43\%)\\
     		28 looked more like sp.1K (7.57\%)  }\\
     		Test result for spanish:\\
     		\texttt{39 looked more like en.1K (10.57\%)\\
     		330 looked more like sp.1K (89.43\%) }
     		
     	\end{enumerate}
     
    \item % problem 4
    	\begin{enumerate}{(b)}
            \item
         \end{enumerate}  

	\item % problem 5
    	\begin{enumerate}
        	\item
        \end{enumerate}
        
\end{enumerate}

\end{document}
